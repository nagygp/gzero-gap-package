% generated by GAPDoc2LaTeX from XML source (Frank Luebeck)
\documentclass[a4paper,11pt]{report}

\usepackage{a4wide}
\sloppy
\pagestyle{myheadings}
\usepackage{amssymb}
\usepackage[latin1]{inputenc}
\usepackage{makeidx}
\makeindex
\usepackage{color}
\definecolor{FireBrick}{rgb}{0.5812,0.0074,0.0083}
\definecolor{RoyalBlue}{rgb}{0.0236,0.0894,0.6179}
\definecolor{RoyalGreen}{rgb}{0.0236,0.6179,0.0894}
\definecolor{RoyalRed}{rgb}{0.6179,0.0236,0.0894}
\definecolor{LightBlue}{rgb}{0.8544,0.9511,1.0000}
\definecolor{Black}{rgb}{0.0,0.0,0.0}

\definecolor{linkColor}{rgb}{0.0,0.0,0.554}
\definecolor{citeColor}{rgb}{0.0,0.0,0.554}
\definecolor{fileColor}{rgb}{0.0,0.0,0.554}
\definecolor{urlColor}{rgb}{0.0,0.0,0.554}
\definecolor{promptColor}{rgb}{0.0,0.0,0.589}
\definecolor{brkpromptColor}{rgb}{0.589,0.0,0.0}
\definecolor{gapinputColor}{rgb}{0.589,0.0,0.0}
\definecolor{gapoutputColor}{rgb}{0.0,0.0,0.0}

%%  for a long time these were red and blue by default,
%%  now black, but keep variables to overwrite
\definecolor{FuncColor}{rgb}{0.0,0.0,0.0}
%% strange name because of pdflatex bug:
\definecolor{Chapter }{rgb}{0.0,0.0,0.0}
\definecolor{DarkOlive}{rgb}{0.1047,0.2412,0.0064}


\usepackage{fancyvrb}

\usepackage{mathptmx,helvet}
\usepackage[T1]{fontenc}
\usepackage{textcomp}
\usepackage{amsmath}


\usepackage[
            pdftex=true,
            bookmarks=true,        
            a4paper=true,
            pdftitle={Written with GAPDoc},
            pdfcreator={LaTeX with hyperref package / GAPDoc},
            colorlinks=true,
            backref=page,
            breaklinks=true,
            linkcolor=linkColor,
            citecolor=citeColor,
            filecolor=fileColor,
            urlcolor=urlColor,
            pdfpagemode={UseNone}, 
           ]{hyperref}

\newcommand{\maintitlesize}{\fontsize{50}{55}\selectfont}

% write page numbers to a .pnr log file for online help
\newwrite\pagenrlog
\immediate\openout\pagenrlog =\jobname.pnr
\immediate\write\pagenrlog{PAGENRS := [}
\newcommand{\logpage}[1]{\protect\write\pagenrlog{#1, \thepage,}}
%% were never documented, give conflicts with some additional packages

\newcommand{\GAP}{\textsf{GAP}}

%% nicer description environments, allows long labels
\usepackage{enumitem}
\setdescription{style=nextline}

%% depth of toc
\setcounter{tocdepth}{1}





%% command for ColorPrompt style examples
\newcommand{\gapprompt}[1]{\color{promptColor}{\bfseries #1}}
\newcommand{\gapbrkprompt}[1]{\color{brkpromptColor}{\bfseries #1}}
\newcommand{\gapinput}[1]{\color{gapinputColor}{#1}}


\begin{document}

\logpage{[ 0, 0, 0 ]}
\begin{titlepage}
\mbox{}\vfill

\begin{center}{\maintitlesize \textbf{The \textsf{GZero} Package\mbox{}}}\\
\vfill

\hypersetup{pdftitle=The \textsf{GZero} Package}
\markright{\scriptsize \mbox{}\hfill The \textsf{GZero} Package \hfill\mbox{}}
{\Huge \textbf{Divisors and Riemann-Roch Spaces of Algebraic Function Fields of Genus Zero\mbox{}}}\\
\vfill

{\Huge Version 0.21\mbox{}}\\[1cm]
{12 October 2017\mbox{}}\\[1cm]
\mbox{}\\[2cm]
{\Large \textbf{ G{\a'a}bor P. Nagy   \mbox{}}}\\
\hypersetup{pdfauthor= G{\a'a}bor P. Nagy   }
\end{center}\vfill

\mbox{}\\
{\mbox{}\\
\small \noindent \textbf{ G{\a'a}bor P. Nagy   }  Email: \href{mailto://nagyg@math.u-szeged.hu} {\texttt{nagyg@math.u-szeged.hu}}\\
  Homepage: \href{http://www.math.u-szeged.hu/~nagyg/} {\texttt{http://www.math.u-szeged.hu/\texttt{\symbol{126}}nagyg/}}}\\
\end{titlepage}

\newpage\setcounter{page}{2}
{\small 
\section*{Copyright}
\logpage{[ 0, 0, 1 ]}
 \index{License} {\copyright} 2017 by G{\a'a}bor P. Nagy

 \textsf{GZero} package is free software; you can redistribute it and/or modify it under the
terms of the \href{http://www.fsf.org/licenses/gpl.html} {GNU General Public License} as published by the Free Software Foundation; either version 2 of the License,
or (at your option) any later version. \mbox{}}\\[1cm]
{\small 
\section*{Acknowledgements}
\logpage{[ 0, 0, 2 ]}
 We appreciate very much all past and future comments, suggestions and
contributions to this package and its documentation provided by \textsf{GAP} users and developers. \mbox{}}\\[1cm]
\newpage

\def\contentsname{Contents\logpage{[ 0, 0, 3 ]}}

\tableofcontents
\newpage

     
\chapter{\textcolor{Chapter }{Introduction}}\label{GZero Introduction}
\logpage{[ 1, 0, 0 ]}
\hyperdef{L}{X7DFB63A97E67C0A1}{}
{
  \index{GZero package} This chapter describes the \textsf{GAP} package \textsf{GZero}. This package implements functionalities for divisors and Riemann-Roch spaces
of an algebraic function field of genus zero. 

 If you are viewing this with on-line help, type: 

 
\begin{Verbatim}[commandchars=!@|,fontsize=\small,frame=single,label=Example]
  !gapprompt@gap>| !gapinput@?GZero package|
\end{Verbatim}
 

 to see the functions provided by the \textsf{GZero} package.

  
\section{\textcolor{Chapter }{Unpacking the \textsf{GZero} Package}}\label{Unpacking the GZero Package}
\logpage{[ 1, 1, 0 ]}
\hyperdef{L}{X7947EB3483D3E6BD}{}
{
  If the \textsf{GZero} package was obtained as a part of the \textsf{GAP} distribution from the ``Download'' section of the \textsf{GAP} website, you may proceed to Section \ref{Compiling Binaries of the GZero Package}. Alternatively, the \textsf{GZero} package may be installed using a separate archive, for example, for an update
or an installation in a non-default location (see  (\textbf{Reference: GAP Root Directories})). 

 Below we describe the installation procedure for the \texttt{.tar.gz} archive format. Installation using other archive formats is performed in a
similar way. 

 To install the \textsf{GZero} package, unpack the archive file, which should have a name of form \texttt{gzero-\mbox{\texttt{\mdseries\slshape XXX}}.tar.gz} for some version number \mbox{\texttt{\mdseries\slshape XXX}}, by typing 

 {\nobreakspace}{\nobreakspace}\texttt{gzip -dc gzero-\mbox{\texttt{\mdseries\slshape XXX}}.tar.gz | tar xpv} 

 It may be unpacked in one of the following locations: 
\begin{itemize}
\item  in the \texttt{pkg} directory of your \textsf{GAP}{\nobreakspace}4 installation; 
\item  or in a directory named \texttt{.gap/pkg} in your home directory (to be added to the \textsf{GAP} root directory unless \textsf{GAP} is started with \texttt{-r} option); 
\item  or in a directory named \texttt{pkg} in another directory of your choice (e.g.{\nobreakspace}in the directory \texttt{mygap} in your home directory). 
\end{itemize}
 In the latter case one one must start \textsf{GAP} with the \texttt{-l} option, e.g.{\nobreakspace}if your private \texttt{pkg} directory is a subdirectory of \texttt{mygap} in your home directory you might type: 

 {\nobreakspace}{\nobreakspace}\texttt{gap -l ";\mbox{\texttt{\mdseries\slshape myhomedir}}/mygap"} 

 where \mbox{\texttt{\mdseries\slshape myhomedir}} is the path to your home directory, which (since \textsf{GAP}{\nobreakspace}4.3) may be replaced by a tilde (the empty path before the
semicolon is filled in by the default path of the \textsf{GAP}{\nobreakspace}4 home directory). }

  
\section{\textcolor{Chapter }{Loading the \textsf{GZero} Package}}\label{Loading the GZero Package}
\logpage{[ 1, 2, 0 ]}
\hyperdef{L}{X8522F06084C32A4D}{}
{
  To use the \textsf{GZero} Package you have to request it explicitly. This is done by calling \texttt{LoadPackage} (\textbf{Reference: LoadPackage}): 

 
\begin{Verbatim}[commandchars=!@|,fontsize=\small,frame=single,label=Example]
  !gapprompt@gap>| !gapinput@LoadPackage("gzero");|
  ----------------------------------------------------------------
  Loading  GZero 0.1
  by G�bor P. Nagy (http://www.math.u-szeged.hu/~nagyg)
  For help, type: ?GZero package 
  ----------------------------------------------------------------
  true
\end{Verbatim}
 

 If \textsf{GAP} cannot find a working binary, the call to \texttt{LoadPackage} will still succeed but a warning is issued informing that the \texttt{HelloWorld()} function will be unavailable. 

 If you want to load the \textsf{GZero} package by default, you can put the \texttt{LoadPackage} command into your \texttt{gaprc} file (see Section{\nobreakspace} (\textbf{Reference: The gap.ini and gaprc files})). }

  
\section{\textcolor{Chapter }{Testing the \textsf{GZero} Package}}\label{Loading the GZero Package}
\logpage{[ 1, 3, 0 ]}
\hyperdef{L}{X868165D87EA0AF48}{}
{
  You can run tests for the package by 
\begin{Verbatim}[commandchars=!@|,fontsize=\small,frame=single,label=Example]
  !gapprompt@gap>| !gapinput@Test(Filename(DirectoriesPackageLibrary("gzero"),"../tst/testall.tst"));|
\end{Verbatim}
 }

 }

        
\chapter{\textcolor{Chapter }{Mathematical background}}\label{GZero Background}
\logpage{[ 2, 0, 0 ]}
\hyperdef{L}{X7EF1B6708069B0C7}{}
{
   
\section{\textcolor{Chapter }{Blabla}}\label{Blabla}
\logpage{[ 2, 1, 0 ]}
\hyperdef{L}{X8120363781D6D51D}{}
{
  Blabla. }

 }

        
\chapter{\textcolor{Chapter }{How to use the package}}\label{GZero Usage}
\logpage{[ 3, 0, 0 ]}
\hyperdef{L}{X7FC3D36E8639C331}{}
{
   
\section{\textcolor{Chapter }{Genus zero curves}}\label{Curves}
\logpage{[ 3, 1, 0 ]}
\hyperdef{L}{X7DA745887BFF189D}{}
{
  The following functions are available: 

\subsection{\textcolor{Chapter }{IsGZ{\textunderscore}Curve}}
\logpage{[ 3, 1, 1 ]}\nobreak
\hyperdef{L}{X80F99CB17BC077A8}{}
{\noindent\textcolor{FuncColor}{$\triangleright$\enspace\texttt{IsGZ{\textunderscore}Curve({\mdseries\slshape obj})\index{IsGZCurve@\texttt{IsGZ{\textunderscore}Curve}}
\label{IsGZCurve}
}\hfill{\scriptsize (Category)}}\\


 A genus zero curve is the projective line over an algebraically closed field. }

 

\subsection{\textcolor{Chapter }{GZ{\textunderscore}Curve}}
\logpage{[ 3, 1, 2 ]}\nobreak
\hyperdef{L}{X8442EE1B792650BA}{}
{\noindent\textcolor{FuncColor}{$\triangleright$\enspace\texttt{GZ{\textunderscore}Curve({\mdseries\slshape K, X})\index{GZCurve@\texttt{GZ{\textunderscore}Curve}}
\label{GZCurve}
}\hfill{\scriptsize (operation)}}\\


 returns the corresponding genus zero divisor over the algebraic closure of the
field \mbox{\texttt{\mdseries\slshape K}}. The indeterminate \mbox{\texttt{\mdseries\slshape X}} generates the corresponding rational function field $K(X)$. }

 

\subsection{\textcolor{Chapter }{IndeterminateOfGZ{\textunderscore}Curve}}
\logpage{[ 3, 1, 3 ]}\nobreak
\hyperdef{L}{X7D6DD3E382EA1965}{}
{\noindent\textcolor{FuncColor}{$\triangleright$\enspace\texttt{IndeterminateOfGZ{\textunderscore}Curve({\mdseries\slshape C})\index{IndeterminateOfGZCurve@\texttt{Indeterminate}\-\texttt{Of}\-\texttt{G}\-\texttt{Z{\textunderscore}}\-\texttt{Curve}}
\label{IndeterminateOfGZCurve}
}\hfill{\scriptsize (function)}}\\


 returns the indeterminate of the function field of the genus zero curve \mbox{\texttt{\mdseries\slshape C}}. }

 

\subsection{\textcolor{Chapter }{UnderlyingField}}
\logpage{[ 3, 1, 4 ]}\nobreak
\hyperdef{L}{X790470C48340E8F7}{}
{\noindent\textcolor{FuncColor}{$\triangleright$\enspace\texttt{UnderlyingField({\mdseries\slshape C})\index{UnderlyingField@\texttt{UnderlyingField}}
\label{UnderlyingField}
}\hfill{\scriptsize (attribute)}}\\


 The underlying field of a genus zero curve is the field of coefficients of the
corresponding algebraic function field. }

 

\subsection{\textcolor{Chapter }{RandomPlaceOfGZ{\textunderscore}Curve}}
\logpage{[ 3, 1, 5 ]}\nobreak
\hyperdef{L}{X847F1C84795287B6}{}
{\noindent\textcolor{FuncColor}{$\triangleright$\enspace\texttt{RandomPlaceOfGZ{\textunderscore}Curve({\mdseries\slshape C})\index{RandomPlaceOfGZCurve@\texttt{Random}\-\texttt{Place}\-\texttt{Of}\-\texttt{G}\-\texttt{Z{\textunderscore}}\-\texttt{Curve}}
\label{RandomPlaceOfGZCurve}
}\hfill{\scriptsize (operation)}}\\
\noindent\textcolor{FuncColor}{$\triangleright$\enspace\texttt{RandomPlaceOfGZ{\textunderscore}Curve({\mdseries\slshape C, d})\index{RandomPlaceOfGZCurve@\texttt{Random}\-\texttt{Place}\-\texttt{Of}\-\texttt{G}\-\texttt{Z{\textunderscore}}\-\texttt{Curve}}
\label{RandomPlaceOfGZCurve}
}\hfill{\scriptsize (operation)}}\\


 returns a random rational place of the genus zero curve \mbox{\texttt{\mdseries\slshape C}}. If the second argument \mbox{\texttt{\mdseries\slshape d}} is given, then it returns a place of degree \mbox{\texttt{\mdseries\slshape d}}. Here, a place is a 1-point divisor of degree one. Notice that the place at
infinity is rational. }

 
\begin{Verbatim}[commandchars=!@|,fontsize=\small,frame=single,label=Example]
  !gapprompt@gap>| !gapinput@Y:=Indeterminate(GF(9),"Y");|
  Y
  !gapprompt@gap>| !gapinput@C:=GZ_Curve(GF(9),Y);|
  <GZ curve over GF(9) with indeterminate Y>
  !gapprompt@gap>| !gapinput@aut:=AutomorphismGroup(C);|
  <group of GZ curve automorphisms of size 720>
  !gapprompt@gap>| !gapinput@Random(aut);|
  GZ_CurveAut([ [ Z(3)^0, Z(3^2)^3 ], [ Z(3^2)^5, Z(3) ] ])
\end{Verbatim}
 

\subsection{\textcolor{Chapter }{FrobeniusAutomorphismOfGZ{\textunderscore}Curve}}
\logpage{[ 3, 1, 6 ]}\nobreak
\hyperdef{L}{X8307ED097B2DB06A}{}
{\noindent\textcolor{FuncColor}{$\triangleright$\enspace\texttt{FrobeniusAutomorphismOfGZ{\textunderscore}Curve({\mdseries\slshape C})\index{FrobeniusAutomorphismOfGZCurve@\texttt{Frobenius}\-\texttt{Automorphism}\-\texttt{Of}\-\texttt{G}\-\texttt{Z{\textunderscore}}\-\texttt{Curve}}
\label{FrobeniusAutomorphismOfGZCurve}
}\hfill{\scriptsize (operation)}}\\


 returns the Frobenius automorphism of the underlying field of the genus zero
curve \mbox{\texttt{\mdseries\slshape C}}. More precisely, the output is an AC-Frobenius automorphism in the sense of
the package \textsf{OnAlgClosure}, acting on the algebraic closure of the underlying finite field. }

 

\subsection{\textcolor{Chapter }{IsGZ{\textunderscore}CurveAutomorphism}}
\logpage{[ 3, 1, 7 ]}\nobreak
\hyperdef{L}{X7B3633597CAFA7AA}{}
{\noindent\textcolor{FuncColor}{$\triangleright$\enspace\texttt{IsGZ{\textunderscore}CurveAutomorphism({\mdseries\slshape obj})\index{IsGZCurveAutomorphism@\texttt{IsG}\-\texttt{Z{\textunderscore}}\-\texttt{Curve}\-\texttt{Automorphism}}
\label{IsGZCurveAutomorphism}
}\hfill{\scriptsize (Category)}}\\


 With automorphisms of an algebraic curve $C$ one means the automorphisms of the corresponding algebraic function field $K(C)$. For genus zero curves over finite fields, the algebraic function field is
the field $K(t)$ of rational functions in one indeterminate. $Aut(K(t))$ consists of fractional linear mappings $t\mapsto \frac{a+bt}{c+dt}$, where $ad-bc\neq 0$. Hence, $Aut(K(t))\cong PGL(2,K)$. 

With fixed Frobenius automorphism $\Phi:x\mapsto x^q$, we can speak of $GF(q)$-rational automorphisms, or, automorphisms defined over $GF(q)$. These form a subgroup isomorphic to $PGL(2,q)$, having a faithful permutation representation of the set $GF(q)\cup \{\infty\}$ of $GF(q)$-rational places. }

 

\subsection{\textcolor{Chapter }{GZ{\textunderscore}CurveAutomorphism}}
\logpage{[ 3, 1, 8 ]}\nobreak
\hyperdef{L}{X7EED9FF48672EFE9}{}
{\noindent\textcolor{FuncColor}{$\triangleright$\enspace\texttt{GZ{\textunderscore}CurveAutomorphism({\mdseries\slshape mat})\index{GZCurveAutomorphism@\texttt{GZ{\textunderscore}}\-\texttt{Curve}\-\texttt{Automorphism}}
\label{GZCurveAutomorphism}
}\hfill{\scriptsize (operation)}}\\
\textbf{\indent Returns:\ }
 the automorphism $t\mapsto \frac{a+bt}{c+dt}$ of the genus zero curve, where \mbox{\texttt{\mdseries\slshape M}} is the nonsingular $2\times 2$ matrix $\begin{pmatrix}a & c\\ b& d\end{pmatrix}$. 

}

 
\subsection{\textcolor{Chapter }{AutomorphismGroup}}\logpage{[ 3, 1, 9 ]}
\hyperdef{L}{X87677B0787B4461A}{}
{
\noindent\textcolor{FuncColor}{$\triangleright$\enspace\texttt{MatrixGroupToGZ{\textunderscore}CurveAutGroup({\mdseries\slshape matgr, C})\index{MatrixGroupToGZCurveAutGroup@\texttt{Matrix}\-\texttt{Group}\-\texttt{To}\-\texttt{G}\-\texttt{Z{\textunderscore}}\-\texttt{Curve}\-\texttt{Aut}\-\texttt{Group}}
\label{MatrixGroupToGZCurveAutGroup}
}\hfill{\scriptsize (function)}}\\
\textbf{\indent Returns:\ }
 the GZ curve automorphism group \$G\$ corresponding to the matrix group \mbox{\texttt{\mdseries\slshape matgr}}. 



 The permutation action of \mbox{\texttt{\mdseries\slshape matgr}} on the set of rational places of \mbox{\texttt{\mdseries\slshape C}} is stored as a nice monomorphism of \$G\$. \noindent\textcolor{FuncColor}{$\triangleright$\enspace\texttt{AutomorphismGroup({\mdseries\slshape C})\index{AutomorphismGroup@\texttt{AutomorphismGroup}}
\label{AutomorphismGroup}
}\hfill{\scriptsize (operation)}}\\
\textbf{\indent Returns:\ }
 the automorphism group of the genus zero curve \mbox{\texttt{\mdseries\slshape C}}. The elements are genus zero automorphisms. The group is isomorphic to $PGL(2,q)$, where $GF(q)$ is the underlying field of \mbox{\texttt{\mdseries\slshape C}}. 

}

 }

 
\section{\textcolor{Chapter }{Genus zero divisors}}\label{Divisors}
\logpage{[ 3, 2, 0 ]}
\hyperdef{L}{X876A3D977D462755}{}
{
  The following functions are available: 

\subsection{\textcolor{Chapter }{IsGZ{\textunderscore}Divisor}}
\logpage{[ 3, 2, 1 ]}\nobreak
\hyperdef{L}{X8338F576793D136C}{}
{\noindent\textcolor{FuncColor}{$\triangleright$\enspace\texttt{IsGZ{\textunderscore}Divisor({\mdseries\slshape obj})\index{IsGZDivisor@\texttt{IsGZ{\textunderscore}Divisor}}
\label{IsGZDivisor}
}\hfill{\scriptsize (Category)}}\\


 A genus zero divisor is a divisor of an algebraic function field of genus 0.
Genus zero divisors form an additive commutative group. }

 

\subsection{\textcolor{Chapter }{GZ{\textunderscore}DivisorConstruct}}
\logpage{[ 3, 2, 2 ]}\nobreak
\hyperdef{L}{X82AC5A5C838D0594}{}
{\noindent\textcolor{FuncColor}{$\triangleright$\enspace\texttt{GZ{\textunderscore}DivisorConstruct({\mdseries\slshape X, pts, ords})\index{GZDivisorConstruct@\texttt{GZ{\textunderscore}DivisorConstruct}}
\label{GZDivisorConstruct}
}\hfill{\scriptsize (function)}}\\


 returns the genus zero divisor over $K(X)$ with points from \mbox{\texttt{\mdseries\slshape pts}} and corresponding orders from \mbox{\texttt{\mdseries\slshape ords}}. $K$ is the prime field of the coefficient field of \mbox{\texttt{\mdseries\slshape X}}. }

 

\subsection{\textcolor{Chapter }{GZ{\textunderscore}Divisor}}
\logpage{[ 3, 2, 3 ]}\nobreak
\hyperdef{L}{X7A5C7B027C47C3ED}{}
{\noindent\textcolor{FuncColor}{$\triangleright$\enspace\texttt{GZ{\textunderscore}Divisor({\mdseries\slshape C, pts, ords})\index{GZDivisor@\texttt{GZ{\textunderscore}Divisor}}
\label{GZDivisor}
}\hfill{\scriptsize (operation)}}\\
\noindent\textcolor{FuncColor}{$\triangleright$\enspace\texttt{GZ{\textunderscore}Divisor({\mdseries\slshape C, pairs})\index{GZDivisor@\texttt{GZ{\textunderscore}Divisor}}
\label{GZDivisor}
}\hfill{\scriptsize (operation)}}\\


 returns the corresponding genus zero divisor over the algebraic function field \mbox{\texttt{\mdseries\slshape F}}. If the indeterminate \mbox{\texttt{\mdseries\slshape X}} is given, then $F=K(X)$, where $K$ is the prime field of the coefficient field of \mbox{\texttt{\mdseries\slshape X}}. }

 

\subsection{\textcolor{Chapter }{GZ{\textunderscore}1PointDivisor}}
\logpage{[ 3, 2, 4 ]}\nobreak
\hyperdef{L}{X874E47E3850CC5DC}{}
{\noindent\textcolor{FuncColor}{$\triangleright$\enspace\texttt{GZ{\textunderscore}1PointDivisor({\mdseries\slshape C, pt})\index{GZ1PointDivisor@\texttt{GZ{\textunderscore}1PointDivisor}}
\label{GZ1PointDivisor}
}\hfill{\scriptsize (operation)}}\\
\noindent\textcolor{FuncColor}{$\triangleright$\enspace\texttt{GZ{\textunderscore}1PointDivisor({\mdseries\slshape C, pt, m})\index{GZ1PointDivisor@\texttt{GZ{\textunderscore}1PointDivisor}}
\label{GZ1PointDivisor}
}\hfill{\scriptsize (operation)}}\\


 returns the zero divisor over the algebraic function field \mbox{\texttt{\mdseries\slshape F}} of genus zero. If the indeterminate \mbox{\texttt{\mdseries\slshape X}} is given, then $F=K(X)$, where $K$ is the prime field of the coefficient field of \mbox{\texttt{\mdseries\slshape X}}. }

 

\subsection{\textcolor{Chapter }{GZ{\textunderscore}ZeroDivisor}}
\logpage{[ 3, 2, 5 ]}\nobreak
\hyperdef{L}{X8200279679BB7FEF}{}
{\noindent\textcolor{FuncColor}{$\triangleright$\enspace\texttt{GZ{\textunderscore}ZeroDivisor({\mdseries\slshape C})\index{GZZeroDivisor@\texttt{GZ{\textunderscore}ZeroDivisor}}
\label{GZZeroDivisor}
}\hfill{\scriptsize (operation)}}\\


 returns the zero divisor over the algebraic function field \mbox{\texttt{\mdseries\slshape F}} of genus zero. If the indeterminate \mbox{\texttt{\mdseries\slshape X}} is given, then $F=K(X)$, where $K$ is the prime field of the coefficient field of \mbox{\texttt{\mdseries\slshape X}}. }

 

\subsection{\textcolor{Chapter }{IsRationalGZ{\textunderscore}Divisor}}
\logpage{[ 3, 2, 6 ]}\nobreak
\hyperdef{L}{X856D9CBB87DC9682}{}
{\noindent\textcolor{FuncColor}{$\triangleright$\enspace\texttt{IsRationalGZ{\textunderscore}Divisor({\mdseries\slshape D})\index{IsRationalGZDivisor@\texttt{IsRational}\-\texttt{G}\-\texttt{Z{\textunderscore}}\-\texttt{Divisor}}
\label{IsRationalGZDivisor}
}\hfill{\scriptsize (attribute)}}\\


 Returns true if \mbox{\texttt{\mdseries\slshape D}} is invariant under the Frobenius automorphism of the underling genus zero
curve. }

 

\subsection{\textcolor{Chapter }{UnderlyingField}}
\logpage{[ 3, 2, 7 ]}\nobreak
\hyperdef{L}{X790470C48340E8F7}{}
{\noindent\textcolor{FuncColor}{$\triangleright$\enspace\texttt{UnderlyingField({\mdseries\slshape D})\index{UnderlyingField@\texttt{UnderlyingField}}
\label{UnderlyingField}
}\hfill{\scriptsize (attribute)}}\\


 The underlying field of a genus zero divisor is the field of coefficients of
the corresponding algebraic function field. }

 

\subsection{\textcolor{Chapter }{Support}}
\logpage{[ 3, 2, 8 ]}\nobreak
\hyperdef{L}{X7B689C0284AC4296}{}
{\noindent\textcolor{FuncColor}{$\triangleright$\enspace\texttt{Support({\mdseries\slshape D})\index{Support@\texttt{Support}}
\label{Support}
}\hfill{\scriptsize (attribute)}}\\


 The support of a genus zero divisor is the set of points with nonzero orders. }

 

\subsection{\textcolor{Chapter }{Valuation}}
\logpage{[ 3, 2, 9 ]}\nobreak
\hyperdef{L}{X80D67BB67A509A56}{}
{\noindent\textcolor{FuncColor}{$\triangleright$\enspace\texttt{Valuation({\mdseries\slshape t, D})\index{Valuation@\texttt{Valuation}}
\label{Valuation}
}\hfill{\scriptsize (operation)}}\\
\noindent\textcolor{FuncColor}{$\triangleright$\enspace\texttt{Valuation({\mdseries\slshape t, ratfun})\index{Valuation@\texttt{Valuation}}
\label{Valuation}
}\hfill{\scriptsize (operation)}}\\


 The valuation of a genus zero divisor $D$ at the point $t$ is its corresponding order. The valuation of a rational function $f=g/h$ at the point \mbox{\texttt{\mdseries\slshape t}} is either the multiplicity of the root \mbox{\texttt{\mdseries\slshape t}} in $g$, or minus the multiplicity of the root \mbox{\texttt{\mdseries\slshape t}} in $h$. If \mbox{\texttt{\mdseries\slshape t}} is $\infty$ then the valuation is $\deg(h)-\deg(g)$. }

 

\subsection{\textcolor{Chapter }{GZ{\textunderscore}PrincipalDivisor}}
\logpage{[ 3, 2, 10 ]}\nobreak
\hyperdef{L}{X8358FD3885FABA30}{}
{\noindent\textcolor{FuncColor}{$\triangleright$\enspace\texttt{GZ{\textunderscore}PrincipalDivisor({\mdseries\slshape C, f})\index{GZPrincipalDivisor@\texttt{GZ{\textunderscore}PrincipalDivisor}}
\label{GZPrincipalDivisor}
}\hfill{\scriptsize (function)}}\\


 returns the principal divisor of the rational function \mbox{\texttt{\mdseries\slshape f}} of the genus zero curve \mbox{\texttt{\mdseries\slshape C}}. }

 

\subsection{\textcolor{Chapter }{GZ{\textunderscore}SupremumDivisor}}
\logpage{[ 3, 2, 11 ]}\nobreak
\hyperdef{L}{X87956889816FD1AF}{}
{\noindent\textcolor{FuncColor}{$\triangleright$\enspace\texttt{GZ{\textunderscore}SupremumDivisor({\mdseries\slshape D1, D2})\index{GZSupremumDivisor@\texttt{GZ{\textunderscore}SupremumDivisor}}
\label{GZSupremumDivisor}
}\hfill{\scriptsize (function)}}\\


 returns the place-wise maximum of the orders of \mbox{\texttt{\mdseries\slshape D1}} and \mbox{\texttt{\mdseries\slshape D2}}. }

 

\subsection{\textcolor{Chapter }{GZ{\textunderscore}InfimumDivisor}}
\logpage{[ 3, 2, 12 ]}\nobreak
\hyperdef{L}{X7FF1448D7B18F5C3}{}
{\noindent\textcolor{FuncColor}{$\triangleright$\enspace\texttt{GZ{\textunderscore}InfimumDivisor({\mdseries\slshape D1, D2})\index{GZInfimumDivisor@\texttt{GZ{\textunderscore}InfimumDivisor}}
\label{GZInfimumDivisor}
}\hfill{\scriptsize (function)}}\\


 returns the place-wise minimum of the orders of \mbox{\texttt{\mdseries\slshape D1}} and \mbox{\texttt{\mdseries\slshape D2}}. }

 

\subsection{\textcolor{Chapter }{GZ{\textunderscore}PositivePartOfDivisor}}
\logpage{[ 3, 2, 13 ]}\nobreak
\hyperdef{L}{X83C3A64A7CBC9FCC}{}
{\noindent\textcolor{FuncColor}{$\triangleright$\enspace\texttt{GZ{\textunderscore}PositivePartOfDivisor({\mdseries\slshape D})\index{GZPositivePartOfDivisor@\texttt{GZ{\textunderscore}}\-\texttt{Positive}\-\texttt{Part}\-\texttt{Of}\-\texttt{Divisor}}
\label{GZPositivePartOfDivisor}
}\hfill{\scriptsize (function)}}\\


 returns the positive part of the divisor \mbox{\texttt{\mdseries\slshape D}}. }

 

\subsection{\textcolor{Chapter }{GZ{\textunderscore}NegativePartOfDivisor}}
\logpage{[ 3, 2, 14 ]}\nobreak
\hyperdef{L}{X81E7F42279A17CA2}{}
{\noindent\textcolor{FuncColor}{$\triangleright$\enspace\texttt{GZ{\textunderscore}NegativePartOfDivisor({\mdseries\slshape D})\index{GZNegativePartOfDivisor@\texttt{GZ{\textunderscore}}\-\texttt{Negative}\-\texttt{Part}\-\texttt{Of}\-\texttt{Divisor}}
\label{GZNegativePartOfDivisor}
}\hfill{\scriptsize (function)}}\\


 returns the negative part of the divisor \mbox{\texttt{\mdseries\slshape D}}. }

 
\begin{Verbatim}[commandchars=!@|,fontsize=\small,frame=single,label=Example]
  !gapprompt@gap>| !gapinput@p1:=GZ_1PointDivisor(C,infinity);|
  <GZ divisor with support of length 1 over indeterminate Y>
  !gapprompt@gap>| !gapinput@p2:=GZ_1PointDivisor(C,Z(3));|
  <GZ divisor with support of length 1 over indeterminate Y>
  !gapprompt@gap>| !gapinput@d:=3*p1-4*p2;|
  <GZ divisor with support of length 2 over indeterminate Y>
  !gapprompt@gap>| !gapinput@Support(d);|
  [ infinity, Z(3) ]
  !gapprompt@gap>| !gapinput@UnderlyingField(d);|
  GF(3^2)
  !gapprompt@gap>| !gapinput@Zero(d);|
  <GZ divisor with support of length 0 over indeterminate Y>
  !gapprompt@gap>| !gapinput@Characteristic(d);|
  3
  gap>
  !gapprompt@gap>| !gapinput@d:=GZ_Divisor(C,[Z(27)^2,Z(3),infinity],[5,-1,2]);|
  <GZ divisor with support of length 3 over indeterminate Y>
  !gapprompt@gap>| !gapinput@Valuation(Z(3),d);|
  -1
  !gapprompt@gap>| !gapinput@Valuation(Z(3)^2,d);|
  0
  gap>
  !gapprompt@gap>| !gapinput@fr:=AC_FrobeniusAutomorphism(9);|
  AC_FrobeniusAutomorphism(3^2)
  !gapprompt@gap>| !gapinput@d^fr;|
  <GZ divisor with support of length 3 over indeterminate Y>
  !gapprompt@gap>| !gapinput@Support(d^fr);|
  [ infinity, Z(3), Z(3^3)^18 ]
  !gapprompt@gap>| !gapinput@Support(d);|
  [ infinity, Z(3), Z(3^3)^2 ]
  gap>
  !gapprompt@gap>| !gapinput@rf:=Y^8-1;|
  Y^8-Z(3)^0
  !gapprompt@gap>| !gapinput@List(GF(9),u->Valuation(u,rf));|
  [ 0, 1, 1, 1, 1, 1, 1, 1, 1 ]
  !gapprompt@gap>| !gapinput@List(GF(9),u->Valuation(u,One(Y)));|
  [ 0, 0, 0, 0, 0, 0, 0, 0, 0 ]
  !gapprompt@gap>| !gapinput@List(GF(9),u->Valuation(u,Zero(Y)));|
  [ -infinity, -infinity, -infinity, -infinity, -infinity, -infinity,
    -infinity, -infinity, -infinity ]
  gap>
  gap>
  !gapprompt@gap>| !gapinput@List(GF(3),u->Valuation(u,One(Y)));|
  [ 0, 0, 0 ]
  !gapprompt@gap>| !gapinput@List(GF(3),u->Valuation(u,Zero(Y)));|
  [ -infinity, -infinity, -infinity ]
\end{Verbatim}
 }

 
\section{\textcolor{Chapter }{Genus zero Riemann-Roch spaces}}\label{RRspaces}
\logpage{[ 3, 3, 0 ]}
\hyperdef{L}{X7F6CA34585EEDA75}{}
{
  

\subsection{\textcolor{Chapter }{GZ{\textunderscore}Equivalent1PointDivisor}}
\logpage{[ 3, 3, 1 ]}\nobreak
\hyperdef{L}{X83EC2B2C801216DD}{}
{\noindent\textcolor{FuncColor}{$\triangleright$\enspace\texttt{GZ{\textunderscore}Equivalent1PointDivisor({\mdseries\slshape D})\index{GZEquivalent1PointDivisor@\texttt{GZ{\textunderscore}}\-\texttt{Equivalent1}\-\texttt{Point}\-\texttt{Divisor}}
\label{GZEquivalent1PointDivisor}
}\hfill{\scriptsize (function)}}\\


 returns the pair $f,m$, where $f$ is a rational function and $m$ is an integer such that $D=Div(f)+m P_{\infty}$. In particular, $D$ is equivalent to the 1-point divisor $mP_{\infty}$. }

 

\subsection{\textcolor{Chapter }{GZ{\textunderscore}RiemannRochSpaceBasis}}
\logpage{[ 3, 3, 2 ]}\nobreak
\hyperdef{L}{X824E0C947B314093}{}
{\noindent\textcolor{FuncColor}{$\triangleright$\enspace\texttt{GZ{\textunderscore}RiemannRochSpaceBasis({\mdseries\slshape D})\index{GZRiemannRochSpaceBasis@\texttt{GZ{\textunderscore}}\-\texttt{Riemann}\-\texttt{Roch}\-\texttt{Space}\-\texttt{Basis}}
\label{GZRiemannRochSpaceBasis}
}\hfill{\scriptsize (function)}}\\


 returns a \textsc{basis} of the Riemann-Roch space of the genus zero divisor \mbox{\texttt{\mdseries\slshape D}}, which is defined by $\{ f \in K[Y] \mid Div(f) \geq - D \}$. }

 
\begin{Verbatim}[commandchars=!@|,fontsize=\small,frame=single,label=Example]
  !gapprompt@gap>| !gapinput@a:=RandomPlaceOfGZ_Curve(C,4);|
  <GZ divisor with support of length 1 over indeterminate Y>
  !gapprompt@gap>| !gapinput@fr:=FrobeniusAutomorphismOfGZ_Curve(C);|
  AC_FrobeniusAutomorphism(3^2)
  !gapprompt@gap>| !gapinput@d:=Sum(AC_FrobeniusAutomorphismOrbit(fr,a));|
  <GZ divisor with support of length 4 over indeterminate Y>
  !gapprompt@gap>| !gapinput@IsRationalGZ_Divisor(d);|
  true
  gap>
  !gapprompt@gap>| !gapinput@GZ_RiemannRochSpaceBasis(3*d);|
  [ Z(3)^0/(Y^12+Y^9+Z(3^2)^2*Y^6+Z(3^2)^3*Y^3+Z(3^2)^2),
    Y/(Y^12+Y^9+Z(3^2)^2*Y^6+Z(3^2)^3*Y^3+Z(3^2)^2),
    Y^2/(Y^12+Y^9+Z(3^2)^2*Y^6+Z(3^2)^3*Y^3+Z(3^2)^2),
    Y^3/(Y^12+Y^9+Z(3^2)^2*Y^6+Z(3^2)^3*Y^3+Z(3^2)^2),
    Y^4/(Y^12+Y^9+Z(3^2)^2*Y^6+Z(3^2)^3*Y^3+Z(3^2)^2),
    Y^5/(Y^12+Y^9+Z(3^2)^2*Y^6+Z(3^2)^3*Y^3+Z(3^2)^2),
    Y^6/(Y^12+Y^9+Z(3^2)^2*Y^6+Z(3^2)^3*Y^3+Z(3^2)^2),
    Y^7/(Y^12+Y^9+Z(3^2)^2*Y^6+Z(3^2)^3*Y^3+Z(3^2)^2),
    Y^8/(Y^12+Y^9+Z(3^2)^2*Y^6+Z(3^2)^3*Y^3+Z(3^2)^2),
    Y^9/(Y^12+Y^9+Z(3^2)^2*Y^6+Z(3^2)^3*Y^3+Z(3^2)^2),
    Y^10/(Y^12+Y^9+Z(3^2)^2*Y^6+Z(3^2)^3*Y^3+Z(3^2)^2),
    Y^11/(Y^12+Y^9+Z(3^2)^2*Y^6+Z(3^2)^3*Y^3+Z(3^2)^2),
    Y^12/(Y^12+Y^9+Z(3^2)^2*Y^6+Z(3^2)^3*Y^3+Z(3^2)^2) ]
  !gapprompt@gap>| !gapinput@ForAll(last,x->x=x^fr);|
  true
\end{Verbatim}
 }

 
\section{\textcolor{Chapter }{Genus zero AG-codes}}\label{AGcodes}
\logpage{[ 3, 4, 0 ]}
\hyperdef{L}{X8519CCF8782192CA}{}
{
  The following functions are available: 

\subsection{\textcolor{Chapter }{IsGZ{\textunderscore}Code}}
\logpage{[ 3, 4, 1 ]}\nobreak
\hyperdef{L}{X810A5F387B8EEA48}{}
{\noindent\textcolor{FuncColor}{$\triangleright$\enspace\texttt{IsGZ{\textunderscore}Code({\mdseries\slshape obj})\index{IsGZCode@\texttt{IsGZ{\textunderscore}Code}}
\label{IsGZCode}
}\hfill{\scriptsize (Category)}}\\
\noindent\textcolor{FuncColor}{$\triangleright$\enspace\texttt{IsGZ{\textunderscore}FunctionalCode({\mdseries\slshape obj})\index{IsGZFunctionalCode@\texttt{IsGZ{\textunderscore}FunctionalCode}}
\label{IsGZFunctionalCode}
}\hfill{\scriptsize (Category)}}\\
\noindent\textcolor{FuncColor}{$\triangleright$\enspace\texttt{IsGZ{\textunderscore}DifferentialCode({\mdseries\slshape obj})\index{IsGZDifferentialCode@\texttt{IsG}\-\texttt{Z{\textunderscore}}\-\texttt{Differential}\-\texttt{Code}}
\label{IsGZDifferentialCode}
}\hfill{\scriptsize (Category)}}\\


 A genus zero code is an algebraic-geometric code defined on an algebraic curve
of genus zero. AG-codes are either of functional or of differential type. }

 

\subsection{\textcolor{Chapter }{GeneratorMatrixOfFunctionalGZ{\textunderscore}CodeNC}}
\logpage{[ 3, 4, 2 ]}\nobreak
\hyperdef{L}{X84E44E038531B3E7}{}
{\noindent\textcolor{FuncColor}{$\triangleright$\enspace\texttt{GeneratorMatrixOfFunctionalGZ{\textunderscore}CodeNC({\mdseries\slshape G, pls})\index{GeneratorMatrixOfFunctionalGZCodeNC@\texttt{Generator}\-\texttt{Matrix}\-\texttt{Of}\-\texttt{Functional}\-\texttt{G}\-\texttt{Z{\textunderscore}}\-\texttt{CodeNC}}
\label{GeneratorMatrixOfFunctionalGZCodeNC}
}\hfill{\scriptsize (function)}}\\


 returns the generator matrix of the functional AG code $C_L(D,G)$, where $D$ is the sum of the degree one places in the list \mbox{\texttt{\mdseries\slshape pls}}. The support of \mbox{\texttt{\mdseries\slshape G}} must be disjoint from \mbox{\texttt{\mdseries\slshape pls}}. }

 

\subsection{\textcolor{Chapter }{GZ{\textunderscore}FunctionalCode}}
\logpage{[ 3, 4, 3 ]}\nobreak
\hyperdef{L}{X7B55263684F49B0B}{}
{\noindent\textcolor{FuncColor}{$\triangleright$\enspace\texttt{GZ{\textunderscore}FunctionalCode({\mdseries\slshape G, D})\index{GZFunctionalCode@\texttt{GZ{\textunderscore}FunctionalCode}}
\label{GZFunctionalCode}
}\hfill{\scriptsize (operation)}}\\
\noindent\textcolor{FuncColor}{$\triangleright$\enspace\texttt{GZ{\textunderscore}FunctionalCode({\mdseries\slshape G})\index{GZFunctionalCode@\texttt{GZ{\textunderscore}FunctionalCode}}
\label{GZFunctionalCode}
}\hfill{\scriptsize (operation)}}\\


 returns the functional AG code $C_L(D,G)=\{(f(P_1),\ldots,f(P_n)) \mid f\in L(G)\}.$ $D$ and $G$ are rational divisors of the genus zero curve $C$. $D=P_1+\cdots+D_n$, where $P_1,\ldots,P_n$ are degree one places of $C$. The supports of $D$ and $G$ are disjoint. If $D$ is not given then it is the sum of affine rational places of $C$. By the Riemann-Roch theorem, functional codes have dimension $\deg(G)+1-g$. }

 

\subsection{\textcolor{Chapter }{GZ{\textunderscore}DifferentialCode}}
\logpage{[ 3, 4, 4 ]}\nobreak
\hyperdef{L}{X7FF603CC86081DD9}{}
{\noindent\textcolor{FuncColor}{$\triangleright$\enspace\texttt{GZ{\textunderscore}DifferentialCode({\mdseries\slshape G, D})\index{GZDifferentialCode@\texttt{GZ{\textunderscore}DifferentialCode}}
\label{GZDifferentialCode}
}\hfill{\scriptsize (operation)}}\\
\noindent\textcolor{FuncColor}{$\triangleright$\enspace\texttt{GZ{\textunderscore}DifferentialCode({\mdseries\slshape G})\index{GZDifferentialCode@\texttt{GZ{\textunderscore}DifferentialCode}}
\label{GZDifferentialCode}
}\hfill{\scriptsize (operation)}}\\


 returns the differential AG code $C_\Omega(D,G) = \{res_{P_1}(\omega),\ldots,res_{P_n}(\omega) \mid \omega \in
\Omega(G-D)\}.$ $D$ and $G$ are rational divisors of the genus zero curve $C$. $D=P_1+\cdots+D_n$, where $P_1,\ldots,P_n$ are degree one places of $C$. The supports of $D$ and $G$ are disjoint. If $D$ is not given then it is the sum of affine rational places of $C$. The differential code is the dual of the corresponding functional code. By
the Riemann-Roch theorem, differential codes have dimension $n-\deg(G)-1+g$. }

 

\subsection{\textcolor{Chapter }{Length}}
\logpage{[ 3, 4, 5 ]}\nobreak
\hyperdef{L}{X780769238600AFD1}{}
{\noindent\textcolor{FuncColor}{$\triangleright$\enspace\texttt{Length({\mdseries\slshape C})\index{Length@\texttt{Length}}
\label{Length}
}\hfill{\scriptsize (attribute)}}\\


 returns the length of the AG code \mbox{\texttt{\mdseries\slshape C}}. }

 

\subsection{\textcolor{Chapter }{GeneratorMatrixOfGZ{\textunderscore}Code}}
\logpage{[ 3, 4, 6 ]}\nobreak
\hyperdef{L}{X784B20AB79F23689}{}
{\noindent\textcolor{FuncColor}{$\triangleright$\enspace\texttt{GeneratorMatrixOfGZ{\textunderscore}Code({\mdseries\slshape C})\index{GeneratorMatrixOfGZCode@\texttt{Generator}\-\texttt{Matrix}\-\texttt{Of}\-\texttt{G}\-\texttt{Z{\textunderscore}}\-\texttt{Code}}
\label{GeneratorMatrixOfGZCode}
}\hfill{\scriptsize (attribute)}}\\


 returns the generator matrix of the AG code \mbox{\texttt{\mdseries\slshape C}} in \textsf{CVEC} matrix format. }

 

\subsection{\textcolor{Chapter }{DesignedMinimumDistance}}
\logpage{[ 3, 4, 7 ]}\nobreak
\hyperdef{L}{X84EB7DAF7DB9DB9F}{}
{\noindent\textcolor{FuncColor}{$\triangleright$\enspace\texttt{DesignedMinimumDistance({\mdseries\slshape C})\index{DesignedMinimumDistance@\texttt{DesignedMinimumDistance}}
\label{DesignedMinimumDistance}
}\hfill{\scriptsize (attribute)}}\\


 returns the designed minimum distance $\delta$ of the genus zero AG code \mbox{\texttt{\mdseries\slshape C}}. When $\deg(G)\geq 2g-2$, then the general formulas for $\delta$ are as follows. For the functional code $C_L(D,G)$, $\delta=n-\deg(G)$, and for the differential code $C_\Omega(D,G)$, $\delta=\deg(G)-(2g-2)$. For genus zero curves, $g=0$ and these formulas give the true minimum distances. }

 
\begin{Verbatim}[commandchars=!@|,fontsize=\small,frame=single,label=Example]
  !gapprompt@gap>| !gapinput@code:=GZ_FunctionalCode(d);|
  <[9,5] genus zero AG-code over GF(3^2)>
  !gapprompt@gap>| !gapinput@Print(code);|
  GZ_FunctionalCode(GZ_Divisor(GZ_Curve(GF(9),Y),
  [ Z(3^8)^302, Z(3^8)^2718, Z(3^8)^3678, Z(3^8)^4782 ],
  [ 1, 1, 1, 1 ]),GZ_Divisor(GZ_Curve(GF(9),Y),
  [ 0*Z(3), Z(3)^0, Z(3), Z(3^2), Z(3^2)^2, Z(3^2)^3, Z(3^2)^5,
    Z(3^2)^6, Z(3^2)^7 ],[ 1, 1, 1, 1, 1, 1, 1, 1, 1 ]))
  !gapprompt@gap>| !gapinput@DesignedMinimumDistance(code);|
  5
\end{Verbatim}
 

\subsection{\textcolor{Chapter }{GZ{\textunderscore}DecodeToCodeword}}
\logpage{[ 3, 4, 8 ]}\nobreak
\hyperdef{L}{X7864B295782524EB}{}
{\noindent\textcolor{FuncColor}{$\triangleright$\enspace\texttt{GZ{\textunderscore}DecodeToCodeword({\mdseries\slshape C, w})\index{GZDecodeToCodeword@\texttt{GZ{\textunderscore}DecodeToCodeword}}
\label{GZDecodeToCodeword}
}\hfill{\scriptsize (operation)}}\\


 Let $\delta$ be the designed minimum distance of \mbox{\texttt{\mdseries\slshape C}}, and define $t=[(\delta-1)/2]$. If there is a codeword $c\in C$ with $d(c,w)\leq t$ then $c$ is returned. Otherwise, the output is \texttt{fail}. 

The decoding algorithm is from [Hoholdt-Pellikaan 1995]. The function \texttt{GZ{\textunderscore}DECODER{\textunderscore}DATA} precomputes two matrices which are stored as attributes of the AG code. The
decoding consists of solving linear equations. }

 
\begin{Verbatim}[commandchars=!@|,fontsize=\small,frame=single,label=Example]
  !gapprompt@gap>| !gapinput@q:=5^3;|
  125
  !gapprompt@gap>| !gapinput@# construct the curve and the divisors|
  !gapprompt@gap>| !gapinput@Y:=Indeterminate(GF(q),"Y");|
  Y
  !gapprompt@gap>| !gapinput@C:=GZ_Curve(GF(q),Y);|
  <GZ curve over GF(125) with indeterminate Y>
  !gapprompt@gap>| !gapinput@P_infty:=GZ_1PointDivisor(C,infinity);|
  <GZ divisor with support of length 1 over indeterminate Y>
  gap>
  !gapprompt@gap>| !gapinput@fr:=FrobeniusAutomorphismOfGZ_Curve(C);|
  AC_FrobeniusAutomorphism(5^3)
  !gapprompt@gap>| !gapinput@P4:=Sum(AC_FrobeniusAutomorphismOrbit(fr,RandomPlaceOfGZ_Curve(C,4)));|
  <GZ divisor with support of length 4 over indeterminate Y>
  !gapprompt@gap>| !gapinput@G:=5*P4+7*P_infty;|
  <GZ divisor with support of length 5 over indeterminate Y>
  !gapprompt@gap>| !gapinput@Degree(G);|
  27
  gap>
  !gapprompt@gap>| !gapinput@len:=90;|
  90
  !gapprompt@gap>| !gapinput@D:=Sum([1..len],i->GZ_1PointDivisor(C,Elements(GF(q))[i]));|
  <GZ divisor with support of length 90 over indeterminate Y>
  gap>
  !gapprompt@gap>| !gapinput@# construct the AG differential code|
  !gapprompt@gap>| !gapinput@agcode:=GZ_DifferentialCode(G,D);|
  <[90,62] genus zero AG-code over GF(5^3)>
  !gapprompt@gap>| !gapinput@DesignedMinimumDistance(agcode);|
  29
  !gapprompt@gap>| !gapinput@Length(agcode)-Degree(G)-1;|
  62
  gap>
  !gapprompt@gap>| !gapinput@# test codeword generation|
  !gapprompt@gap>| !gapinput@t:=Int((DesignedMinimumDistance(agcode)-1)/2);|
  14
  !gapprompt@gap>| !gapinput@sent:=Random(agcode);;|
  !gapprompt@gap>| !gapinput@err:=RandomVectorOfGivenWeight(GF(q),Length(agcode),t);;|
  !gapprompt@gap>| !gapinput@received:=sent+err;;|
  gap>
  !gapprompt@gap>| !gapinput@# decoding|
  !gapprompt@gap>| !gapinput@sent_decoded:=GZ_DecodeToCodeword(agcode,received);|
  <cvec over GF(5,3) of length 90>
  !gapprompt@gap>| !gapinput@sent=sent_decoded;|
  true
\end{Verbatim}
 }

 
\section{\textcolor{Chapter }{Utilities for genus zero AG-codes}}\label{Utilities}
\logpage{[ 3, 5, 0 ]}
\hyperdef{L}{X7ACE2DF587589A08}{}
{
  

\subsection{\textcolor{Chapter }{RestrictVectorSpace}}
\logpage{[ 3, 5, 1 ]}\nobreak
\hyperdef{L}{X853BF1BC7F6462B8}{}
{\noindent\textcolor{FuncColor}{$\triangleright$\enspace\texttt{RestrictVectorSpace({\mdseries\slshape V, F})\index{RestrictVectorSpace@\texttt{RestrictVectorSpace}}
\label{RestrictVectorSpace}
}\hfill{\scriptsize (function)}}\\


 Let $K$ be a field and $V$ a linear subspace of $K^n$. The restriction of \mbox{\texttt{\mdseries\slshape V}} to the field \mbox{\texttt{\mdseries\slshape F}} is the intersection $V\cap F^n$. }

 

\subsection{\textcolor{Chapter }{UPolCoeffsToSmallFieldNC}}
\logpage{[ 3, 5, 2 ]}\nobreak
\hyperdef{L}{X8596810487A72298}{}
{\noindent\textcolor{FuncColor}{$\triangleright$\enspace\texttt{UPolCoeffsToSmallFieldNC({\mdseries\slshape f, q})\index{UPolCoeffsToSmallFieldNC@\texttt{UPolCoeffsToSmallFieldNC}}
\label{UPolCoeffsToSmallFieldNC}
}\hfill{\scriptsize (function)}}\\


 This non-checking function returns the same polynomial as \mbox{\texttt{\mdseries\slshape f}}, making sure that the coefficients are in $GF(q)$. }

 

\subsection{\textcolor{Chapter }{RandomVectorOfGivenWeight}}
\logpage{[ 3, 5, 3 ]}\nobreak
\hyperdef{L}{X7D6980057AF64327}{}
{\noindent\textcolor{FuncColor}{$\triangleright$\enspace\texttt{RandomVectorOfGivenWeight({\mdseries\slshape F, n, k})\index{RandomVectorOfGivenWeight@\texttt{RandomVectorOfGivenWeight}}
\label{RandomVectorOfGivenWeight}
}\hfill{\scriptsize (function)}}\\


 returns a random vector of $F^n$ of Hamming weight $k$. \noindent\textcolor{FuncColor}{$\triangleright$\enspace\texttt{RandomVectorOfGivenDensity({\mdseries\slshape F, n, delta})\index{RandomVectorOfGivenDensity@\texttt{RandomVectorOfGivenDensity}}
\label{RandomVectorOfGivenDensity}
}\hfill{\scriptsize (function)}}\\


 returns a random vector of $F^n$ in which the density of nonzero elements is approxiamtively $\delta$. \noindent\textcolor{FuncColor}{$\triangleright$\enspace\texttt{RandomBinaryVectorOfGivenWeight({\mdseries\slshape n, k})\index{RandomBinaryVectorOfGivenWeight@\texttt{RandomBinaryVectorOfGivenWeight}}
\label{RandomBinaryVectorOfGivenWeight}
}\hfill{\scriptsize (function)}}\\


 returns a random vector of $GF(2)^n$ of Hamming weight $k$. \noindent\textcolor{FuncColor}{$\triangleright$\enspace\texttt{RandomBinaryVectorOfGivenDensity({\mdseries\slshape n, delta})\index{RandomBinaryVectorOfGivenDensity@\texttt{RandomBinaryVectorOfGivenDensity}}
\label{RandomBinaryVectorOfGivenDensity}
}\hfill{\scriptsize (function)}}\\


 returns a random vector of $GF(2)^n$ in which the density of nonzero elements is approxiamtively $\delta$. }

 }

 }

        
\chapter{\textcolor{Chapter }{An example: BCH codes as genus zero AG-codes}}\label{GZero Example}
\logpage{[ 4, 0, 0 ]}
\hyperdef{L}{X8683ED827FE8F859}{}
{
   The following example constructs BCH codes as genus zero AG-codes. 
\begin{Verbatim}[commandchars=!@|,fontsize=\small,frame=single,label=Example]
  !gapprompt@gap>| !gapinput@my_BCH:=function(n,l,delta,F)|
  !gapprompt@>| !gapinput@	local q,m,r,s,beta,Y,C,D_beta,P_0,P_infty,agcode;|
  !gapprompt@>| !gapinput@	#|
  !gapprompt@>| !gapinput@	q:=Size(F);|
  !gapprompt@>| !gapinput@	m:=OrderMod(q,n);|
  !gapprompt@>| !gapinput@	beta:=Z(q^m)^((q^m-1)/n);|
  !gapprompt@>| !gapinput@	#|
  !gapprompt@>| !gapinput@	Y:=Indeterminate(F,"Y");|
  !gapprompt@>| !gapinput@	C:=GZ_Curve(GF(q^m),Y);|
  !gapprompt@>| !gapinput@	D_beta:=Sum([0..n-1],i->GZ_1PointDivisor(C,beta^i));|
  !gapprompt@>| !gapinput@	P_0:=GZ_1PointDivisor(C,0);|
  !gapprompt@>| !gapinput@	P_infty:=GZ_1PointDivisor(C,infinity);|
  !gapprompt@>| !gapinput@	#|
  !gapprompt@>| !gapinput@	r:=l-1;|
  !gapprompt@>| !gapinput@	s:=n+1-delta-l;|
  !gapprompt@>| !gapinput@	agcode:=GZ_FunctionalCode(r*P_0+s*P_infty,D_beta);|
  !gapprompt@>| !gapinput@	#|
  !gapprompt@>| !gapinput@	return RestrictVectorSpace(agcode,F);|
  !gapprompt@>| !gapinput@end;|
  function( n, l, delta, F ) ... end
  !gapprompt@gap>| !gapinput@|
  !gapprompt@gap>| !gapinput@####|
  !gapprompt@gap>| !gapinput@|
  !gapprompt@gap>| !gapinput@q:=2;|
  2
  !gapprompt@gap>| !gapinput@n:=35;|
  35
  !gapprompt@gap>| !gapinput@l:=1;|
  1
  !gapprompt@gap>| !gapinput@delta:=5;|
  5
  !gapprompt@gap>| !gapinput@|
  !gapprompt@gap>| !gapinput@|
  !gapprompt@gap>| !gapinput@C0:=BCHCode(n,l,delta,GF(q)); time;|
  a cyclic [35,11,5]8..13 BCH code, delta=5, b=1 over GF(2)
  24
  !gapprompt@gap>| !gapinput@C1:=my_BCH(n,l,delta,GF(q)); time;|
  <vector space over GF(2), with 11 generators>
  364
  !gapprompt@gap>| !gapinput@|
  !gapprompt@gap>| !gapinput@Collected(List(C0,x->Number(x,y->IsOne(y))));|
  [ [ 0, 1 ], [ 5, 7 ], [ 7, 5 ], [ 10, 56 ], [ 13, 105 ], [ 14, 10 ], 
    [ 15, 105 ], [ 16, 385 ], [ 17, 350 ], [ 18, 350 ], [ 19, 385 ], 
    [ 20, 105 ], [ 21, 10 ], [ 22, 105 ], [ 25, 56 ], [ 28, 5 ], 
    [ 30, 7 ], [ 35, 1 ] ]
  !gapprompt@gap>| !gapinput@Collected(List(C1,x->Number(x,y->IsOne(y))));|
  [ [ 0, 1 ], [ 5, 7 ], [ 7, 5 ], [ 10, 56 ], [ 13, 105 ], [ 14, 10 ], 
    [ 15, 105 ], [ 16, 385 ], [ 17, 350 ], [ 18, 350 ], [ 19, 385 ], 
    [ 20, 105 ], [ 21, 10 ], [ 22, 105 ], [ 25, 56 ], [ 28, 5 ], 
    [ 30, 7 ], [ 35, 1 ] ]
  !gapprompt@gap>| !gapinput@|
  !gapprompt@gap>| !gapinput@SetDesignedMinimumDistance(C1,delta);|
  !gapprompt@gap>| !gapinput@DesignedMinimumDistance(C1);|
  5
\end{Verbatim}
 }

    \def\indexname{Index\logpage{[ "Ind", 0, 0 ]}
\hyperdef{L}{X83A0356F839C696F}{}
}

\cleardoublepage
\phantomsection
\addcontentsline{toc}{chapter}{Index}


\printindex

\newpage
\immediate\write\pagenrlog{["End"], \arabic{page}];}
\immediate\closeout\pagenrlog
\end{document}
